\documentclass[a4j, 11pt]{jarticle}
% START:共通設定&共通パッケージ読み込み(基本変更しない)
\renewcommand{\baselinestretch}{1.4}
\setlength{\oddsidemargin}{-0mm}
\setlength{\textwidth}{16cm}
\setlength{\topmargin}{-1.5cm}
\setlength{\textheight}{24cm}
\setlength{\baselineskip}{2cm}
\special{pdf: minorversion=7}    % 出力するPDFのバージョンを指定
\usepackage{ifthen}              % if文制御用
\usepackage[dvipdfmx]{graphicx}
\usepackage{amsmath}             % 数式用
\usepackage{array}               % 数式での場合分け用
\usepackage{url}                 % URL表示用
\usepackage{here}                % [H]用
\usepackage[dvipdfmx]{hyperref}  % 全体像把握&簡易移動のため
\usepackage{pxjahyper}           % 日本語のしおり(ブックマーク)表示用
\hypersetup{pdfborder = {0 0 0}} % hyperrefリンクの囲みを消す
\pagenumbering{roman}            % ページ番号をアラビア数字に変更
\newcounter{fiscal_year}         % 卒業年度計算用
\setcounter{fiscal_year}{\the\year}
\ifthenelse{\the\month < 4}{
	% 年明けから3月までは年-1にする
	\addtocounter{fiscal_year}{-1}
}{}
% END:共通設定&共通パッケージ読み込み(基本変更しない)


% START:ユーザ設定&ユーザパッケージ読み込み---------


% END:ユーザ設定&ユーザパッケージ読み込み-----------


\begin{document}
% START:タイトル
\begin{titlepage}\Large ~
{\normalsize \the\value{fiscal_year} 年度卒業}
\vfill
\begin{center}

% START: 論文の種類-------------------------------
%{\Huge 修士論文}
{\Huge 卒業論文}
% END: 論文の種類---------------------------------
\end{center}
\begin{center}

% START: 日本語タイトル---------------------------
深層学習を用いたロバストな白線補完
% END: 日本語タイトル-----------------------------
\end{center}
\begin{center}

% START: 英語タイトル-----------------------------
English Title
% END: 英語タイトル-------------------------------
\end{center}
\vfill
\begin{center}
\begin{tabular}{|c|l|}
\hline

% START: 論文の種類-------------------------------
%所属 & 新潟大学自然科学研究科 電気情報工学専攻・林隆史研究室 \\
所属 & 新潟大学工学部情報工学科・林隆史研究室 \\
% END: 論文の種類---------------------------------
\hline

% START: 在籍番号---------------------------------
在籍番号 & T16I273C \\
% END: 在籍番号-----------------------------------
\hline

% START: 論文著者---------------------------------
氏名 & 小松 耀人 \\
% START: 論文著者---------------------------------
\hline
\end{tabular}
\end{center}
\vspace{1cm}
\vfill
\end{titlepage}
\pagebreak
\addtocounter{page}{1}
\thispagestyle{empty}  % このページにページ番号を振らない
% END:タイトル

% START:アブストラクト-----------------------------
\section*{概要}
 近年高齢者による事故が増え自動運転への期待も高まっている.しかしながら現在の研究で最も高い精度を出している自動運転技術は高精度地図を必要とし,高精度地図が整備されている道路は先進国全体でも1\%に満たない.高精度地図を全道路に整備するのは作成に人の手を介する点や,一般道路は更新頻度も高いという点で非現実的である. ゆえに,出発地から目的地まで人の手を一切介さない完全な自動運転の実現には高精度地図に依存せず,自車の周りの状況を把握し,その情報をもとに適切な進路をリアルタイムに選択することが不可欠である.特に路面上の情報を画像などから計測し,自車が走行すべき車線を計算することは、乗車している人の安全を保証する意味で非常に重要である.しかし,一般道路には白線が途切れていたりかすれていたりする場所が散見され,現在の技術では線がない部分の認識はできず,それらが道路状況を把握することの障害となっている. そこで本研究では敵対的生成ネットワーク(GAN)を用いて,白線にかすれやることによって,白線の途切れやカスレを自動補完する手法を提案する.
\section*{Abstract}
English Abstract Here

% END:アブストラクト-------------------------------

% START:目次作成
\newpage
\tableofcontents       % 目次作成
\thispagestyle{empty}  % このページにページ番号を振らない
\pagebreak
\pagenumbering{arabic} % ページ番号をアラビア数字に変更
% END:目次作成


% START:本編--------------------------------------
\section{はじめに}
近年高齢者による事故が増えているが、地方では車がないと生活が困難な地域もあり免許を返納してもらうことが非現実的な地域も多くある。また、健常者が移動手段として使う車でも事故のリスクを少しでも減らし安全な社会を実現すべく運転支援技術の高度化や出発地から目的地まで一切人の手を介することのない完全な自動運転の実現が急がれている。\\
\indent 現在の研究で最も完全な自動運転に近い運転支援を実現しているのは通常のカーナビゲーションシステムに使用されている2Dの俯瞰図ではなく、数センチメートル以内の誤差しかない超高精度の3D地図にさらに道路上の構造物、車線情報、路面情報などの「静的情報」、道路工事やイベントなどによる交通規制などの「準静的情報」、信号の現示情報などの「動的情報」、観測時点の渋滞状況などの「準動的情報」などを付加したものでこの地図が整備されているのは先進国全体で1\%未満である。このような地図の整備状況が未だに狭い理由としては、そもそもの作成に人力でCADを用いた細かい調整が必要なことに加え、一般道路は更新の頻度も高く新しい道路がどんどんと増えていくことや地点ごとの車線や路面の整備状況の差異が非常に大きく手動で逐一更新していくのは非現実的である。様々な機関が自動的に高精度地図を整備する技術を提案しているが、未だに一般道路を自動的に高精度な地図に起こすまでには至っていない。\cite{tri-ad}\cite{autolab}\\
\indent このような背景から自車の周囲の車線や路面の情報を観測し、活用していくことが完全な自動運転の実現には必要不可欠である。また、仮に高精度地図が普及したとしても搭乗者の安全を守り正確な自動運転を行うために自車からの周辺観測は非常に重要である。\\
\indent 現在製品化している運転支援技術は先行車がいてかつ高速道路上で限られた速度帯でのみなど非常に限られた状況でしか作動せず、完全な自動運転には程遠い。限られた状況でしか作動しない原因の一つに一般道路は高速道路などの高規格幹線道路とは異なり道路整備の頻度も低く舗装や塗装の基準も異なるので場所によって白線や横断歩道などの路面上の情報が不完全になっているものも多くこれらを含む路面情報を認識するには前処理として本来必要であるのに欠落している情報を補完する必要がある。\cite{road_design}\\
\indent 以上の点を踏まえて本研究では深層学習を用いた画像補完技術を用いて本来必要であるのに白線が消えたりかすれたりしている部分を補完することを目的とする。画像補完にはGenerative Adversarial Network(GAN)を用い、ニューラルネットワークを用いて白線があるべき領域を重点的に学習し、補完を試みる。\\
\indent 各章の説明(あとから)

% END:本編----------------------------------------

% START:参考文献----------------------------------
%% ベタ打ちの場合
\newpage
\begin{thebibliography}{1}
	\bibitem{tri-ad}TRI-AD\\ \url{https://www.tri-ad.global/jp/news/20190425} (2020年01月27日アクセス) 
	\bibitem{autolab}自動運転LAB\\ \url{https://jidounten-lab.com/u_autonomous-map-company} (2020年01月27日アクセス)
	\bibitem{road_design}第6章 道路舗装に関する設計基準\\ \url{http://www.mlit.go.jp/sogoseisaku/inter/keizai/gijyutu/pdf/road_design_j2.pdf} (2020年01月27日アクセス)
	\bibitem{key1}サイト名\\ \url{http://google.com} (yyyy年mm月dd日アクセス)
	\bibitem{key1}サイト名\\ \url{http://google.com} (yyyy年mm月dd日アクセス) % ウェブサイトの場合
\bibitem{key2}著者,書籍タイトル,出版                                      % 書籍,論文の場合
\end{thebibliography}


%% bibtexを使用する場合
%\bibliography{ref}         % .bibファイルから拡張子を外した名前 ex)ref.bib
%\bibliographystyle{junsrt} % 参考文献出力スタイル
%\nocite{*}                 % 参照していない項目も出力する
% END:参考文献------------------------------------


\newpage
\section*{謝辞}
本研究を進めるにあたり,ご指導を頂いた林隆史教授に厚く感謝申し上げます.
また,日常の議論を通じて多くの知識や示唆を頂いた林隆史研究室の皆様に感謝いたします.

\end{document}
%------------------------------------------------
