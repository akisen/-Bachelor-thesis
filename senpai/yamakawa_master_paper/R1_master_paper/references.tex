%----------------------------------------------------------------------
% 修士論文スタイルファイルのサンプル
% 修士論文:(タイトル未定)
% 新潟大学 大学院自然科学研究科 情報・計算機専攻 情報ネットワーク講座
% 博士前期課程
% 丸山秀志(maruyama@net.ie.niigata-u.ac.jp)
% 提出予定日:(提出日未定)
%----------------------------------------------------------------------

%----------------------------------------------------------------------
%プリアンブルの指定です.
%11ptでjreportクラスを使用しています.
%ですから,一番大きな見出しがchapterになります.
%注意してください.
%----------------------------------------------------------------------
%---数字を○で囲む時に使用---
\def\MARU#1{\leavevmode \setbox0\hbox{$\bigcirc$}%
\copy0\kern-\wd0 \hbox to\wd0{\hfil{#1}\hfil}}

%\usepackage{showkeys}%推敲用
\pagestyle{plain}
\makeatletter
%\def\@cite#1{$\m@th^{\hbox{\@ove@rcfont #1)}}$}
\def\@cite#1{\m@th{\hbox{[#1]}}}
\makeatother

%----------------------------------------------------------------------
%参考文献の記入欄です.
%----------------------------------------------------------------------
\bibliographystyle{junsrt}
\bibliography{pdga-jun}
\addcontentsline{toc}{chapter}{参考文献}
\begin{thebibliography}{99}% 文献数が10未満の時 {9}

\bibitem{Mingwen}
Mingwen Dong, “Convolutional Neural Network Achieves Human-level Accuracy in Music Genre Classification, ” Feb. 27, 2018.

\bibitem{gradcam}
Ramprasaath R. Selvaraju, Michael Cogswell, Abhishek Das, Ramakrishna Vedantam, Devi Parikh, Dhruv Batra, ”Grad-CAM Visual Explanations from Deep Networks via Gradient-based Localizations, ” March 21 2017.

\bibitem{Adam}
Diederik P. Kingma, Jimmy Lei Ba, ”ADAM: A Method for Stochastic Optimizer, ” Jan. 30, 2017

\bibitem{CNN}
”畳み込みニューラルネットワーク,” https://ml4a.github.io/ml4a/jp/convnets/, 参照Dec. 12. 2019.

\bibitem{GAN}
”GANと損失関数の計算についてまとめた,” \\
https://qiita.com/kzkadc/items/f49718dc8aedbe8a1bee, 参照Dec. 12. 2019.

\bibitem{wgan}
Martin Arjovsky. Soumith Chintala, Leon Bottou, ”Wasserstein GAN, ” Dec. 6, 2017.

\bibitem{wgan-gp}
Ishaan Gulrajani, Faruk Ahmed, Martin Arjovsky, Vincent Dumoulin Aaron Courville, ”Improved Training of Wasserstein GANs, ” Dec. 25, 2017

\bibitem{MINE}
Mohamed Ishmael Belghazi, Aristide Baratin, Sai Rajeswar, Sherjil Ozair, Yoshua Bengio, Aaron Courville, R Devon Hjelm, ”Mutual Information Neural Estimation, ” June 7, 2018.

\bibitem{deepinfomax}
R Devon Hjelm, Alex Fedorov, Samuel Lavoie-Marchildon, Karan Grewal, Phil Bachman, Adam Trischler, Yoshua Bengio, ”Leaning Deep Representations by Mutual Information Estimation and Maximization, ” Feb. 22, 2019.

\bibitem{melspect}
”メル尺度,” https://ja.wikipedia.org/wiki/メル尺度, 参照Dec. 12. 2019.

\bibitem{gtzan}
”MARSYAS, Music Analysis Retrieval and SYnthesis for Audio Signals, http://marsyas.info/downloads/datasets.html, Sep. 2018

\bibitem{gtzanissue}
Bob L. Sturm, ”The GTZAN dataset: Its contents, its faults, their effects on evaluation, and its future use, ”June 10, 2013

\bibitem{percuss_harmony}
Derry FitzGerald, ”Hamonic/Percussive Separation Using Median Filtering, ”Sep. 6-10, 2010, Austria.

\bibitem{weibin}
Weibin Zhang, Wenkang Lei, Xiangmin Xu, Xiaofeng Xing, ”Improved Music Genre Classification with Convolutional Neural Network, ”Sep. 8-12, 2016, SanFrancisco, USA.

\bibitem{genres}
”音楽のジャンル一覧,” https://ja.wikipedia.org/wiki/音楽のジャンル一覧, 参照Dec. 12. 2019

\end{thebibliography}