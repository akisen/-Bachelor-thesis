\chapter{おわりに}
本研究では,特徴量を自動で抽出する深層学習を用いて楽曲ジャンルの分類を行った.さらに学習して得られた分類モデルから,楽曲ジャンルの境界となる部分の可視化する手法を提案した.


楽曲データとしては10ジャンルを持つGTZANデータセットを用いて評価を行った.このとき,提案手法では重複のあるデータの削除を行うことでより精度評価の妥当性を確保した.


分類モデルを構築する際には,分類精度で高い評価を受けているCNNを用いた.ここでの入力をパーカッション成分とハーモニー成分に分けたメル周波数スペクトログラムを正規化したデータを用いることにより,従来よりも分類精度を向上させることができた.さらに10分割交差検証を行ったところ,学習データの組み合わせによって精度に偏りが出やすいという事が分かった.

次に構築した分類モデルとGANと組み合わせることにより,ノイズベクトルからメル周波数スペクトログラムの生成を行った.生成されるスペクトログラムの音楽ジャンルにおいて,DiscoとHip-hopは低周波に強めの一定のビートが特徴として表れやすいことが確認できた.また,生成されるスペクトログラムは連続変化が可能なため,分類モデルのジャンル出力確率も連続で変化することができるモデルとなった.


さらにGANの入力である2次のノイズベクトルの値とジャンル分類結果の関係を2次元ジャンルマップ空間に示した.マップ内の値を連続変化させたときの出力確率の変化をグラフで表したところ,ジャンルの確率変化が連続であることを確認でき,ジャンル境界となる部分の可視化を可能にし,提案手法のジャンル分類器の評価にも用いた.


今後の課題としては,連続変化するジャンル確率とスペクトログラムの関係性の分析を行っていきたいと考えている.