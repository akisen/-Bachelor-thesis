\chapter{はじめに}
インターネットの普及に伴い,様々なマルチメディアデバイスが増加する中,音楽コンテンツのディジタル化も劇的に増加している.ユーザの嗜好に合わせたより良い楽曲を提示するためにも,自動で楽曲を特徴づけるアルゴリズムないしは方法・枠組みといったものが必要不可欠である.この課題は音楽情報検索(Musical Information Retrieval : MIR)という研究テーマとして様々な事例が挙げられており,その中の一つとして楽曲ジャンル分類がある.楽曲ジャンルの分類基準は一般に曖昧かつ不明瞭であるため,明確なアルゴリズムで自動的に分類することはとても困難である.

このような背景の下,近年では機械学習を用いて楽曲ジャンルを分類を行う研究が注目されている.機械学習を用いたモデルは,分類する基準を統計的に最適化していくため,明確なアルゴリズムを作ることが困難であるパターン認識において優れた手法となっている.さらに機械学習の一種である深層学習においては,データの特徴量を自動で抽出するという特長を有し,非線形な分類問題にも対応できることから,複雑な問題の近似解を求めるための手法として様々な分野で活用されている.こうした理由から,深層学習を楽曲ジャンル分類に適用した研究も行われており,一例としてMingwenらによる畳み込みニューラルネットワークを用いたものが挙げられる\cite{Mingwen}.

深層学習はエンドツーエンドな学習でモデル設計が比較的簡単な一方,学習した中身がブラックボックスであり,判断根拠が不明という欠点がある.楽曲ジャンル分類を例にとるならば,どういった音色や音の大きさがジャンル分類に寄与しているかがわからないことになる.これは誤った判別をした際にどのようにモデルを修正すればよいかといった問題が挙げられる.そのため,人間に解釈可能な表現をあてはめることによってモデルの信頼性を高めるといったことも重要であり,深層学習の判断根拠を定量化する研究も行われている.その中でも,分類時の判断根拠となるデータ箇所をヒートマップ化をして可視化を行うという観点で,Grad-CAMという手法がある\cite{gradcam}.しかしながら,この方法は学習済みモデルにおける特徴マップの出力サイズに大きく依存してしまい,モデル設計の仕方によっては意味のないヒートマップを作ってしまうことがある.


以上の点を踏まえて本研究では,深層学習を用いることによって,楽曲ジャンルという分類基準が不明瞭なデータを自動分類し,学習した分類モデルからジャンル境界を可視化する新規手法を提案することを目的とする.分類モデルはパターン認識に優れたConvolutional Neural Network (CNN)を用いることによって分類精度の向上図る.また,CNNで得られた学習済み分類モデルとGenerative Adversarial Network (GAN)を組み合わせることによって,CNNが分類するためのデータを生成する.これによりGANへの入力ノイズベクトルと生成データの分類結果の関係を2次元ジャンルマップとして可視化をする.


第2章では機械学習の一つであるニューラルネットを用いた学習法の全体概要を説明した後,分類モデルでよく用いられるCNNと,生成モデルで用いられるGANについて細かく説明する.さらにニューラルネットを用いた相互情報量の推定と最大化をする手法についても説明する.
第3章ではモデルの入力として用いるメル周波数スペクトログラムについて説明する.第4章ではCNNを用いた楽曲ジャンル分類の関連研究として\cite{Mingwen}と可視化手法である\cite{gradcam}の手法を説明し,それぞれの問題点を述べる.また,周波数スペクトログラムをハーモニー成分とパーカッション成分に分ける手法を説明する.
第5章では提案手法について述べる.はじめに学習用データセットについて述べ,CNNを用いて楽曲ジャンルを分類するモデルを構築する.次に学習済みCNNとGANを用いて,CNNが分類するための入力データを生成するモデルを構築する.最後に生成モデルを用いて2次元ジャンルマップの可視化を行う手法を提案する.
第6章では,提案手法における分類モデルと生成モデルの評価を行い,作成した2次元ジャンルマップに対し考察を行う.
